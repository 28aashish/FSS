\section{Sparse Matrix Data Structures}
Sparse matrices are characterized by a relatively few non-zero elements.
This property can be leveraged to store and operate these matrices efficiently. 
The three most popular sparse matrix storage formats are \textit{Triplet Format}, 
\textit{Compressed Column Sparse} and \textit{Compressed Row Sparse}.
Both time and memory space complexity off any sparse matrix operations are highly 
dependent on the underlying storage format and hence it is very important to 
carefully select the storage format depending on the operations to be performed.
\\
\\
\textbf{Triplet Format}\\
Triplet format is the simplest sparse matrix data structure consisting 
lists of all the non-zero entries in the matrix and corresponding row and Column
indices.
\\
\\
\textbf{Compressed Column Sparse}\\
Teh CCS format consists of 3 arrays. 1) an array of non-zero entries with all the 
elements in the same column are listed one after the other 2) an array row indices 
corresponding to all the non-zero elements and 3) an array of the pointers
where each column starts.
\\
\\
\textbf{Compressed Row Sparse}\\
CRS is similar to the CCS structure. It consists a array of the row pointers instead of 
the column pointers.
\\
\\
\textbf{ELLPACK}\\
The ELLPACK is a block based storage format and consists of two matrices, a data matrix
and column index matrix. The size of both matrices is determined by the number of 
rows and the maximum number of number of non-zero elements in all the rows.
Additional zeros are appended in a row to have uniform row length. This increases the storage overhead and can be
eliminated with the advanced format such as sliced ELLPACK format.
\\
\\
The triplet format is simple to create but difficult to use in most sparse matrix
algorithms because of the absence of ordering of the elements. CCS is the most 
widely used format because of many sparse algorithms uses column frontend to proceed.
Many sparse matrix libraries requires the row/column indices to appear in a ascending 
order for the ease of access and simplicity, although it is non necessary.


\begin{figure}
    \centering
    \begin{subfigure}[b]{0.45\textwidth}
        \centering
        $\begin{bmatrix}
            5 &  0 & -5 &  0 & 6 \\
            0 &  4 &  0 & -4 & 0 \\
            2 &  0 &  0 &  0 & 0 \\
            1 & -3 &  0 & -1 & 0 \\
            0 &  0 & -2 &  0 & 3 \\
        \end{bmatrix}$
        \caption{Example Matrix}
        \label{fig:sparseDS:example}
    \end{subfigure}
    \begin{subfigure}[b]{0.5\textwidth}
        \centering
        \resizebox{\textwidth}{!}{\begin{tabular}{|c|c|c|c|c|c|c|c|c|c|c|c|}
            \hline
            Values & 5 & -5 & 6 & 4 & -4 & 2 & - & -3 & -1 & -2 & 3 \\
            \hline
            Column Indices & 0 & 2 & 4 & 1 & 3 & 0 & 0 & 1 & 3 & 2 & 4 \\
            \hline
            Row Indices & 0 & 0 & 0 & 1 & 1 & 2 & 3 & 3 & 3 & 4 & 4 \\
            \hline
        \end{tabular}}
        \caption{Triplet format}
        \label{fig:sparseDS:triplet}
    \end{subfigure}
    \begin{subfigure}[b]{0.49\textwidth}
        \centering
        \resizebox{\textwidth}{!}{\begin{tabular}{|c|c|c|c|c|c|c|c|c|c|c|c|}
            \hline
            Values & 5 & 2 & 1 & 4 & -3 & -5 &-2 & -4 & -1 & 6 & 3 \\
            \hline
            Row Indices & 0 & 2 & 3 & 1 & 3 & 0 & 4 & 1 & 3 & 0 & 4 \\
            \hline
            Column Pointers & \multicolumn{3}{c|}{0} & \multicolumn{2}{c|}{3} & \multicolumn{2}{c|}{5} & \multicolumn{2}{c|}{7} & \multicolumn{2}{c|}{9} \\
            \hline
        \end{tabular}}
        \caption{Compresses Column Sparse}
        \label{fig:sparseDS:ccs}
    \end{subfigure}
    \begin{subfigure}[b]{0.49\textwidth}
        \centering
        \resizebox{\textwidth}{!}{\begin{tabular}{|c|c|c|c|c|c|c|c|c|c|c|c|}
            \hline
            Values & 5 & -5 & 6 & 4 & -4 & 2 & - & -3 & -1 & -2 & 3 \\
            \hline
            Column Indices & 0 & 2 & 4 & 1 & 3 & 0 & 0 & 1 & 3 & 2 & 4 \\
            \hline
            Row Pointers & \multicolumn{3}{c|}{0} & \multicolumn{2}{c|}{3} & \multicolumn{2}{c|}{5} & \multicolumn{2}{c|}{6} & \multicolumn{2}{c|}{9} \\
            \hline
        \end{tabular}}
        \caption{Compresses Row Sparse}
        \label{fig:sparseDS:ccs}
    \end{subfigure}
    \caption{Storage formats for sparse matrices}
    \label{fig:sparseDS:formatsExample}
\end{figure}