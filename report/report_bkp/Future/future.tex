\chapter{Future Work}
\vspace{-1.5cm}
\textbf{Synthesizable Quad Port BRAM}\\
Though the design of the quad port BRAM is correct, the synthesis tool
detects the false condition of hold violation between two registers triggered by 
two different clocks. This hold violation is then propagated to the 
implementation stage and the layout tool fails to meet timing requirements. Even if the 
hold condition is invalid the setup condition for the second register must be satisfied and hence we can not
report the path as a false path. A proper solution must be found for this problem to be able to use
time multiplexed BRAMs.
\\

\textbf{Integration with the On-Chip Processor}\\
The current hardware implementation requires a wrapper testbench to feed the instruction and 
input data. Porting the scheduler tool chain to the on-chip ARM core of the Zynq SoC
can eliminate the need to generate additional BRAM coefficient files inputs and instructions.
Also the results can be redirected easily because of the shared memory scape of the 
processor and FPGA programable logic.
\\

\textbf{Column Group Based Schedule}\\
Extracting fine grin parallelism provides significant gain over the coarse-grain solution
but at the cost of pre-processing time. The level wise scheduling policy used by Nechma \cite{Nechma}
is also interesting as it generates the schedule for each column one by one. We can define
an intermediate approach where instead of looking at the entire graph we can generate a schedule
for group of columns. The groups can be formed based on the level of the column node in the 
symbolic analysis.
\\

\textbf{Automatic Generation of the Time-Multiplexed Multiplexers}\\
The current implementation generates multiplexers without adding intermediate 
registers to optimize the path delays. This is fine because of the smaller number
of the processing elements used in this project. In order to increase number of processing
units or BRAMs multiplexers must be pipelined. The path delay of the pipeline
stage has to determined based on the target FPGA. 
