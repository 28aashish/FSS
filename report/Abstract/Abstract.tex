\begin{Abstract}
\noindent
Solving the sparse linear system is one of the most critical steps in many scientific applications such as circuit simulation, training of neural networks, power system modeling, and 5G communication. These operations are iterative and majorly consist of sparse form. In such scenarios, it becomes essential to develop a more efficient way to solve the equations using graph algorithms instead of traditional techniques like Gaussian elimination. The project presents a scalable FPGA-based LU solver system geared towards the matrices that arise in circuit simulations. The LU decomposition approach specified in this project has three main parts. The first part does a symbolic analysis of the matrix. The structure of the sparse system remains the same during the entire simulation and hence can be analyzed symbolically only once to generate a directed acyclic graph. The second part takes this directed flow graph and hardware features such as the number of arithmetic units and BRAMs as inputs and generates a static schedule using the priority list-based ASAP strategy for cross-bar type network. The final software toolchain is a hardware implementation of the cross-bar network.The design was Packed over AXI Protocoal and integrated with Microblaze( Xilinx intellectual property Processor ) and SHAKTI ( open-source initiative SoCs ).

The Digital Physical IP was implemented with frequency of 40MHz on SkyWater Open Source PDK on 130nm technology.The design include SRAM created using OpenRAM,open-source SRAM compiler, and the Processing elements were implemented using OpenLane,OpenLane is an automated RTL to GDSII flow based ASIC implementation. The macro integration was done with the help of Cadence SoC Encounter with satifying generic DRCs, placements, antenna effect and timing constraints.

The future aspect is to achieve better timing performance,Scalable Architecture and implement DFT without design errors.


% \afterpage{\blankpage}
% \clearpage
\end{Abstract}